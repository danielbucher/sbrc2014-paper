\documentclass[12pt]{article}

\usepackage{sbc-template}
\usepackage{graphicx,url}
\usepackage[brazil]{babel}
\usepackage[utf8]{inputenc}
\usepackage{float}
\usepackage{setspace}

\usepackage{tabularx}
\usepackage{cite}

\sloppy

\title{Title}

\author {Daniel C. Bucher\inst{1}, Rodrigo Campiolo\inst{1,2}, Daniel M. Batista\inst{1} }


\address{Instituto de Matemática e Estatística -- Universidade de São Paulo (USP)\\
  Rua do Matão, 1010 - CEP 05508-090 - São Paulo - SP
\nextinstitute
  Universidade Tecnológica Federal do Paraná (UTFPR)\\
  Campo Mourão, PR -- Brasil
  \email{\{dbucher, batista\}@ime.usp.br, rcampiolo@utfpr.edu.br}
}

\begin{document}

\maketitle

\begin{resumo}
  Resumo.

\textbf{Palavras chave:} keywords, go, here.

\end{resumo}

\begin{abstract}
  Abstract.

\textbf{Palavras chave:} keywords, go, here.

\end{abstract}

%-----------------------------------------------------------------------------%
\section{Introdução} \label{sec:intro}

Introdução será escrita por último.

%-----------------------------------------------------------------------------%
\section{Trabalhos Relacionados} \label{sec:rel}

%TODO Internet Relay Chat deveria permanecer em itálico?
Nesta seção, apresentamos alguns trabalhos relacionados. Os trabalhos são
divididos em dois tipos. O primeiro consiste em trabalhos que procuram usar fontes de
dados abertas para detectar alertas de problemas de segurança em redes de
computadores. O segundo tipo consiste em trabalhos que buscam realizar algum
tipo de investigação, geralmente em busca de atividades criminosas, em canais do
\textit{Internet Relay Chat} (IRC).

\cite{santos2013} apresentaram um método de extração de notificações de
segurança de redes de computadores a partir de mensagens postadas na rede de
\textit{microblogging}, Twitter.
%
A metodologia apresentada nesse trabalho, consiste em coletar mensagens
postadas no Twitter e processar essas mensagens usando as técnicas de
agrupamento e classificação. Na seção \ref{sec:metod} descrevemos essas duas
técnicas, uma vez que utilizaremos as mesmas neste trabalho.
%
A a combinação dessas técnicas obteve uma precisão de 92\% de mensagens
que continham tópicos relacionados à segurança de redes, e 50\% que
representavam potenciais alertas.

No que diz respeito ao IRC, a maioria dos trabalhos encontrados por nós
realizam uma análise \textit{post mortem} de logs contendo as mensagens de
determinado canal.
%
Existem poucos trabalhos que buscam fazer algum tipo de análise em tempo real,
ou quase em tempo real, de mensagens na rede.
%
Em sua dissertação de mestrado, \cite{brown2007} propõe uma arquitetura para
uma ferramenta de investigação automática no IRC. A arquitetura proposta
consiste em cinco módulos: (\textit{i}) o módulo de coleta responsável por
capturar as mensagens e realizar um \textit{parse} em busca de datas, nomes
de usuário e \textit{hyperlinks};
%
(\textit{ii}) o módulo de armazenamento
salva as mensagens e eventos, como entrada e saída de usuários em canais, em
um banco de dados;
%
(\textit{iii}) o módulo de análise verifica se existem referências à atividades
criminosas nas mensagens;
%
(\textit{iv}) o módulo de alerta recupera os dados do módulo de análise e envia,
por correio eletrônico e mensagens de texto no celular, para as autoridades responsáveis;
%
e por fim, (\textit{v}) o módulo localizador tenta rastrear a origem dos
usuários remetentes das mensagens. Para tanto, ele utiliza o comando \textbf{WHOIS} do
protocolo do IRC. No entanto, Brown avisa que esse módulo pode não
retornar informação relevante.
%
A análise das mensagens é realizada utilizando as técnicas de análise de
palavras chave e análise de bancos de dados \cite{brown2007}.
%
Essa abordagem, no entanto, requer que o investigador saiba com antecedência
quais canais monitorar.

\cite{michels2012} propôs uma ferramenta de análise em tempo real de
conversas no IRC. Nesse trabalho difere da abordagem dos demais trabalhos
encontrados na literatura pois adiciona um módulo de \textit{crawlers} para
auxiliar o investigador a encontrar os canais a serem monitorados.
%
Esse módulo funciona em duas etapas. A primeira consiste no que ele chama de
\textit{topic crawler}, ele realiza a análise de palavras chave nos nomes e
nas descrições de todos os canais de determinado servidor. A seguir, um
\textit{channel crawler} entra nos canais identificados e escuta as conversas
por um curto período de tempo e realiza a análise de palavras chaves de forma
a identificar com maior precisão se o canal deve ser marcado para revisão do
investigador.

A análise das mensagens foi realizada de duas formas. Primeiramente era
realizada tanto a análise de palavras chave por si só, quanto a análise de
palavras chave combinada com a análise de \textbf{\textit{Part of Speech
Tagging} (POST)} \cite{stanford}. A análise POST quebra a frase em partes e
define as classes gramaticais de cada parte. No entanto, Michels
observou que o uso desse tipo de análise não produz diferenças significativas,
se comparado com o uso de análise de palavras chave sozinho.

O segundo tipo de análise foi a análise de tópico quase em tempo real.
Michels implementou o método de análise \textit{classic frequency}
\cite{gainaru2010} pois ele é facilmente adaptável para uma ferramenta que analisa
dados em tempo real.
%
Michels ressalta que no geral, análises de tópico são realizadas em logs
após a coleta dos dados. Para simular esse efeito, ele definiu um limite de
mensagens de forma que a análise de tópicos só era realizada após o módulo de
coleta ter coletado esse limite.
%
A conclusão foi de que o tópico de análise não funciona bem porque as
ferramentas utilizadas para \textit{stemming} (processo de reduzir uma
palavra derivada à palavra raiz que a originou), \textit{thesaurus}, etc.
não estão preparadas para analisar jargões e gírios, comuns quando se trata
de comunicação através da internet.

Na seção \ref{sec:metod}, apresentamos a metodologia adotada por nós, que
é uma combinação da metodologia utilizada por \cite{santos2013} e
\cite{michels2012}.
%
Descartamos o uso da análise \textbf{POST} devido à complexidade de adicionar
a mesma na ferramenta proposta por nós em relação ao baixo benefício observado
por Michels.



%-----------------------------------------------------------------------------%
\section{Metodologia} \label{sec:metod}

Escreva a Metodologia aqui. (usamos apenas canais que não pedem senha)

%-----------------------------------------------------------------------------%
\section{Resultados} \label{sec:res}

Escreva os resultados aqui.

%-----------------------------------------------------------------------------%
\section{Discussão} \label{sec:disc}

Escreva os resultados aqui.


%-----------------------------------------------------------------------------%
\section{Conclusão e Trabalhos Futuros} \label{sec: concl}

Escreva a conclusão assim.

\bibliographystyle{sbc}
\bibliography{paper}
\end{document}
