%%%%%%%%%%%%%%%%%%%%%%%%%%%%%%%%%%%%%%%%%
% Beamer Presentation
% LaTeX Template
% Version 1.0 (10/11/12)
%
% This template has been downloaded from:
% http://www.LaTeXTemplates.com
%
% License:
% CC BY-NC-SA 3.0 (http://creativecommons.org/licenses/by-nc-sa/3.0/)
%
%%%%%%%%%%%%%%%%%%%%%%%%%%%%%%%%%%%%%%%%%

%----------------------------------------------------------------------------------------
%	PACKAGES AND THEMES
%----------------------------------------------------------------------------------------

\documentclass{beamer}

\mode<presentation> {

% The Beamer class comes with a number of default slide themes
% which change the colors and layouts of slides. Below this is a list
% of all the themes, uncomment each in turn to see what they look like.

%\usetheme{default}
%\usetheme{AnnArbor}
%\usetheme{Antibes}
%\usetheme{Bergen}
%\usetheme{Berkeley}
%\usetheme{Berlin}
%\usetheme{Boadilla}
%\usetheme{CambridgeUS}
%\usetheme{Copenhagen}
%\usetheme{Darmstadt}
%\usetheme{Dresden}
%\usetheme{Frankfurt}
%\usetheme{Goettingen}
%\usetheme{Hannover}
%\usetheme{Ilmenau}
%\usetheme{JuanLesPins}
%\usetheme{Luebeck}
\usetheme{Madrid}
%\usetheme{Malmoe}
%\usetheme{Marburg}
%\usetheme{Montpellier}
%\usetheme{PaloAlto}
%\usetheme{Pittsburgh}
%\usetheme{Rochester}
%\usetheme{Singapore}
%\usetheme{Szeged}
%\usetheme{Warsaw}

% As well as themes, the Beamer class has a number of color themes
% for any slide theme. Uncomment each of these in turn to see how it
% changes the colors of your current slide theme.

%\usecolortheme{albatross}
%\usecolortheme{beaver}
%\usecolortheme{beetle}
%\usecolortheme{crane}
%\usecolortheme{dolphin}
%\usecolortheme{dove}
%\usecolortheme{fly}
%\usecolortheme{lily}
%\usecolortheme{orchid}
%\usecolortheme{rose}
%\usecolortheme{seagull}
%\usecolortheme{seahorse}
%\usecolortheme{whale}
%\usecolortheme{wolverine}

%\setbeamertemplate{footline} % To remove the footer line in all slides uncomment this line
%\setbeamertemplate{footline}[page number] % To replace the footer line in all slides with a simple slide count uncomment this line

%\setbeamertemplate{navigation symbols}{} % To remove the navigation symbols from the bottom of all slides uncomment this line
}

\usepackage{graphicx} % Allows including images
\usepackage{booktabs} % Allows the use of \toprule, \midrule and \bottomrule in tables
\usepackage[utf8]{inputenc}

%----------------------------------------------------------------------------------------
%	TITLE PAGE
%----------------------------------------------------------------------------------------

\title[AP2]{Network Security Alerts Detection on Internet Relay Chat Network}

\author{Daniel Bucher} % Your name
\institute[USP] % Your institution as it will appear on the bottom of every slide, may be shorthand to save space
{
Universidade de São Paulo \\ % Your institution for the title page
\medskip
\textit{dbucher@ime.usp.br} % Your email address
}
\date{\today} % Date, can be changed to a custom date

\begin{document}

\begin{frame}
\titlepage % Print the title page as the first slide
\end{frame}

\begin{frame}
\frametitle{Overview}

\begin{enumerate}
    \item{Introduction}
    \item{Objectives}
    \item{Work Plan}
    \item{Schedule}
    \item{Methods}
    \item{Result Analysis}
    \item{References}
\end{enumerate}
\end{frame}

%----------------------------------------------------------------------------------------
%	PRESENTATION SLIDES
%----------------------------------------------------------------------------------------
\begin{frame}
\frametitle{Introduction}

The internet provides a large source of information on various areas.
Tools that automatically collect and analyse those can help decrease the time
users take to research a topic. This is particularly important when administrat-
ing computer networks.
\end{frame}

%----------------------------------------------------------------------------------------

\begin{frame}
\frametitle{Objectives}

\begin{itemize}
    \item Develop of a method for extracting network security information from Internet
        Relay Chat networks.
    \item Monitor a number of channels on the course of one month and analyse resulting
        messages to determine if they contain valid information.
\end{itemize}

Research question:

\textbf{Q1} Does IRC various channels provide valuable information that can help detect
network security threats?
\end{frame}

%----------------------------------------------------------------------------------------

\begin{frame}
\frametitle{Work Plan}

\textbf{Tasks: }

\begin{enumerate}
    \item{Develop message extractor.}
    \item{Develop crawlers.}
    \item{Select channels.}
    \item{Extract messages.}
    \item{Improve filters.}
    \item{Improve grouping module.}
    \item{Improve the visibility module.}
    \item{Determine if messages extracted from IRC channels are meaningful.}
\end{enumerate}

\end{frame}

%----------------------------------------------------------------------------------------

\begin{frame}
\frametitle{Schedule}
\begin{center}
    \begin{table}
        \resizebox{\linewidth}{!}{
        \begin{tabular}{ | c | c | c | c | c | c | c | c | c | c | }
            \hline
            \textbf{Task} & \textbf{Week 1} & \textbf{Week 2} & \textbf{Week 3} &  %
            \textbf{Week 4} & \textbf{Week 5} & \textbf{Week 6} & \textbf{Week 7} & %
            \textbf{Week 8} & \textbf{Week 9}\\
            \hline
            Task 01 & X & X & X &   &   &   &   &   &  \\
            \hline
            Task 02 &   & X & X &   &   &   &   &   &  \\
            \hline
            Task 03 &   &   & X & X &   &   &   &   &  \\
            \hline
            Task 04 &   &   & X & X & X & X & X &   &  \\
            \hline
            Task 05 &   &   &   &   & X & X &   &   &  \\
            \hline
            Task 06 &   &   &   &   &   & X & X &   &  \\
            \hline
            Task 07 &   &   &   &   &   &   & X & X &  \\
            \hline
            Task 08 &   &   &   &   &   &   &   & X & X\\
            \hline
        \end{tabular}}
        \caption{Schedule.}
        \label{table:schedule}
    \end{table}
\end{center}
\end{frame}

%----------------------------------------------------------------------------------------

\begin{frame}
\frametitle{Methods}

\begin{enumerate}
    \item{Crawling: topic crawler and channel crawler.}
    \item{Monitoring: bots monitor selected channels.}
    \item{Filtering.}
    \item{Similarity grouping.}
    \item{Visibility.}
\end{enumerate}

\end{frame}

%----------------------------------------------------------------------------------------

\begin{frame}
\frametitle{Result Analysis}

\begin{itemize}
    \item{Number of important messages.}
    \item{Manual analysis of messages to verify existance of important information.}
\end{itemize}
\end{frame}

%------------------------------------------------

\begin{frame}
\frametitle{Thank you!}

\Huge{\centerline{Questions?}}
\end{frame}

%----------------------------------------------------------------------------------------

\begin{frame}
\frametitle{References}
\footnotesize{
\begin{thebibliography}{99} % Beamer does not support BibTeX so references must be inserted manually as below
\bibitem[Campiolo et. al., 2013]{p1} L. A. F. Santos and R. Campiolo and M. A. Gerosa and D. M. Batista (2013)
\newblock Detecção de Alertas de Segurança em Redes de Computadores Usando Redes Sociais
\newblock \emph{31º Simpósio Brasileiro de Redes de Computadores e Sistemas Distribuídos}

\bibitem[Michels, 2013]{p2} M. O. Michels (2013)
\newblock Real Time Text Analysis on Internet Relay Chat Conversations
\newblock \emph{Center for Education and Research Information Assurance and Security, Purdue University}

\end{thebibliography}
}
\end{frame}

%----------------------------------------------------------------------------------------

\end{document}
