\documentclass[12pt]{article}

\usepackage{sbc-template}
\usepackage{graphicx,url}
%\usepackage[brazil]{babel}
\usepackage[utf8]{inputenc}
\usepackage{float}
\usepackage{setspace}

\usepackage{tabularx}
\usepackage{cite}

\begin{document}
\sloppy

\title{Network Security Alerts Detection on Internet Relay Chat Network}

\author {Daniel C. Bucher\inst{1}}


\address{Instituto de Matemática e Estatística -- Universidade de São Paulo (USP)\\
  Rua do Matão, 1010 - CEP 05508-090 - São Paulo - SP
  \email{dbucher@ime.usp.br}
}


\maketitle
\begin{abstract}
  The internet provides a large source of information on various areas. Tools
  that automatically collect and analyse those can help decrease the time
  users take to research a topic.
  %
  This is particularly important when administrating computer networks.
  %
  The proposal of this work is to determine if the Internet Relay Chat network
  can provide usefull information concerning network security.
  %
  In order to answer that question, we propose the development of a method for
  extracting network security information from Internet Relay Chat networks.
  %
  We will also monitor a number of channels on the course of one month and analyse
  resulting messages to determine if they contain valid information.

\textbf{Keywords:} Internet Relay Chat, network security, keyword analysis.

\end{abstract}


%-----------------------------------------------------------------------------%
\section{Introduction} \label{sec:intro}

It is important for network administrators to have access to recently
discovered security threats in order to keep up with measures to keep their
networks secured. This access also helps organizations that develop security
solutions to keep their tools up to date with recent threats.
%
With that in mind, \cite{campiolo2013} came with a methodology to extract
posts with relevant information of network security threats on Twitter.
As a result, they obtained and efficient method that presents only tweets
with the related topic.

But Twitter is just one of the many sources that the internet provide that
can contaign usefull information on this area.
%
Internet Relay Chat, or IRC, has withstood the test of time and is still
largely used as a communication channel within software communities and
hacker groups.
%
In that context, we believe that the IRC network can provide valuable
information concerning network security threats.

The Internet Relay Chat is a application layer protocol\footnote{%
The original IRC RFC: \url{https://tools.ietf.org/html/rfc1459#section-1.1}}
that allows a group of users to transfer messages on defined channels.
%
Channels are named groups of one or more clients. When a client sends a
message to a channel, all other clients on the same channel receive that
message.
%It was created on the
lates 1980s and is still
used today as one of the main, multi-users, chat services \cite{michels2013}.
%
IRC is a client-server protocol that uses TCP and optionally, TLS for
encryption.
%
IRC servers can connect to other servers and form an IRC network.

On his master thesis, \cite{michels2013} presented an IRC Data Gathering Tool which
facilitated real-time analysis of IRC chat messages. His goal was to detect
criminal activities going on on the IRC network.
%
Altough that is not our objective, \cite{michels2013} did prove that is possible
to analyze in near real-time IRC chat messages.
%
In section \ref{sec:methods} we describe a methodology based on both the work
of \cite{campiolo2013} and \cite{michels2013}.

This paper is divided in five sections. On section \ref{sec:objectives} we
present the objectives and research question. On section \ref{sec:workplan}
we present a work plan and a schedule. On section \ref{sec:methods} we present
both material and method to be used. Section \ref{sec:result} presents how we
plan to analyse the results obtained.

%-----------------------------------------------------------------------------%
\section{Objectives} \label{sec:objectives}

The objective to this project is to answer the folowing question:

\begin{description}
    \item{\textbf{Q1} Does IRC various channels provide valuable information
        that can help detect network security threats?}
\end{description}

In order to answer that question, we propose developping an application that
autommatically detects channels that are candidate to provide information
in network security threats and extract that information.
%
This application will be developed in the form os modules that will be
incorporated on the framework developt by \cite{campiolo2013}.
%
A visualization model will also be avaiable so that interested personnel can
have acces to that information.
%
In section \ref{sec:methods} we present the method to be used to detect and
extract this messages.
%
After analysing the resulting messages, we can determine if this method
did result in a set of messages with valuable information.

This proposal also has the objective of increasing the source of information
of the project developed by \cite{campiolo2013}, that proposed a similar method
and framework that extracts network security threats posts on Twitter.

%-----------------------------------------------------------------------------%
\section{Work Plan and Schedule} \label{sec:workplan}

In this section we present the work plan and schedule for this project. The
tasks that will be performed in order to accomplish the objectives described on
section \ref{sec:objectives} are:

\begin{enumerate}
    \item{Develop message extractor.}
    \item{Develop crawlers.}
    \item{Select channels.}
    \item{Extract messages.}
    \item{Improve filters.}
    \item{Improve grouping module.}
    \item{Improve the visibility module.}
    \item{Determine if messages extracted from IRC channels are meaningful.}
\end{enumerate}

Table \ref{table:schedule} presents the schedule proposed for the project. Week
1 starts on September 22, and week 9 on November's 17. The
following subsections explains in details some of the above tasks.

\begin{center}
    \begin{table}
        \resizebox{\linewidth}{!}{
        \begin{tabular}{ | c | c | c | c | c | c | c | c | c | c | }
            \hline
            \textbf{Task} & \textbf{Week 1} & \textbf{Week 2} & \textbf{Week 3} &  %
            \textbf{Week 4} & \textbf{Week 5} & \textbf{Week 6} & \textbf{Week 7} & %
            \textbf{Week 8} & \textbf{Week 9}\\
            \hline
            Task 01 & X & X & X &   &   &   &   &   &  \\
            \hline
            Task 02 &   & X & X &   &   &   &   &   &  \\
            \hline
            Task 03 &   &   & X & X &   &   &   &   &  \\
            \hline
            Task 04 &   &   & X & X & X & X & X &   &  \\
            \hline
            Task 05 &   &   &   &   & X & X &   &   &  \\
            \hline
            Task 06 &   &   &   &   &   & X & X &   &  \\
            \hline
            Task 07 &   &   &   &   &   &   & X & X &  \\
            \hline
            Task 08 &   &   &   &   &   &   &   & X & X\\
            \hline
        \end{tabular}}
        \caption{Schedule.}
        \label{table:schedule}
    \end{table}
\end{center}

\subsection{Message Extractor} \label{subsec:extractor}

The purpose of this module is to extract messages from channels on the IRC
network. This is the first task that'll be performed on the project since it
is essential to the achievment of it's goals.
%
Here we will use the ruby gem, Cinch, to extract and persist messages from
IRC channels. By performing this task, we will be able to acknowledge which
information is possible to retrieve from IRC messages.

By the end of week 3, we expect to be able to retrieve messages that posses
the keywords that we will define on the channels selected by the crawlers
described on the nex subsection.

\subsection{Crawlers and Selecting Channels}

IRC networks are composed of various servers, which if many channels that
users can join and chat. In order to analyze messages, we first need to
select a group of channels that we're interested in monitoring.
%
For this, we'll develop a couple of crawlers that will go through channels
searching for keywords, both on the description and on user messages, and
select the channels that are candidate of containing network security threats
messages.
%
The next section describes this crawling process.

Extracting messages and tasks related to filter and grouping will follow
the methodoly defined by \cite{campiolo2013}, with some modifications on the
relevancy decision since IRC messages aren't limited to 160 characteres as
Twitter's posts.


%-----------------------------------------------------------------------------%
\section{Material and Methods} \label{sec:methods}

On this section we present the material and methods to be used to develop the
solution being proposed.
%
The material to be used consists of a dedicated virtual machine that'll be
used for hosting the application to be developed. This machine will run a
Linux based operating system and needs to be connected to the internet.
%
The framework to be used to extract messages from the IRC network is a
Ruby gem called Cinch\footnote{Cinch homepage: %
\url{https://github.com/cinchrb/cinch}}.
%
For the outputs of the application, we will use CSV and YAML formats with
the extracted messages and additional information.

The methodology we propose is divided into an crawling stage, an extraction
stage, a filtering stage, a grouping stage and a visibility stage.
% Crawling stage
In order to extract messages from IRC, we first need to select which servers
and channels we'll monitor. With that in mind, we propose the use of a
metodology similar to what \cite{michels2013} did on his master
thesis. The method consist of using a Crawler module, composed of two crawlers,
to detect channels that are worth monitoring.
%
The first Crawler, the Topic Crawler, goes through the listing of each channel
and perform a keyword analysis on their description. As a result, we get a list
of channels that possibly address network security topics.
%
The second Crawler, the Channel Crawler, does a similar work on the channels
resulting from the previous listing, but oberserving conversations instead of
just reading the description. The idea is to have se Channel Crawler analysing
messages for a short period of time in order to detect conversations that
address the topic of network security. If a keyword is detected, the channel
will be flagged and the crawler continues to the next.
%
This two Crawling process will result in a listing of interesting
servers and channels to be monitored on the following stages. This list
is to be reviewed mannualy in order to remove channels judged unninteresting.

% Monitoring stage
Next comes the monitoring stage. On this stage, each channel resulting from the
crawling stage will have a bot monitoring all conversation within it while
looking for a set of keywords (e.g., security, virus, exploit). Messages
captured will be stored on a database. The data model to be used is still
to be decided.
% Other stages
The other stages, such as filtering and grouping will be adaptations of the
architecture presented by \cite{campiolo2013}. On the next section we present
how we plan to analyse the results to be obtained.

%-----------------------------------------------------------------------------%
\section{Result Analysis} \label{sec:result}

In order to analyze if the methods described on section \ref{sec:methods}
answer the research question presented on section \ref{sec:objectives}, we'll
first take a look at the number of messages defined as important on the end of
the process.
%
If our filters don't capture any message on the time period of our experiment,
we'll assume that IRC doesn't provide a good source of network security threats
related information.

Next, we'll analyse some of the messages defined as important on the end of the
process to see if they contain relevant information concerning network security
threats.
%
We expect to find a goog number of messages with relevant information on them.

\bibliographystyle{sbc}
\bibliography{paper}
\end{document}
